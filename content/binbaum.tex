Wir haben die Zufallsvariablen $\mathcal X_1,\mathcal X_2 $ für die Augenzahl zweier fairer Würfel mit 
\[E(\mathcal X_1)=E(\mathcal X_2)=3,5 \]
\[E(\max(\mathcal X_1,\mathcal X_2))\approx4,47 \]
\[=\sum pr(Z=z) z = \frac{1}{36}1+\frac{3}{36}2+\frac{5}{36}3+\hdots \]
\textbf{Frage:}\\
Was ist die erwartete maximale Tiefe eines naturwüchsigen binären Baumes?

$t_n$ zugehörige Zufallsvariable
\[E(T_n)=\sum_{i=1}^n \frac{1}{n}E(\max(t_{i-1},T_{n-i})+1\]
\fbox{\parbox{\columnwidth}{ Nebenrechnung:\small
\begin{align*}
\max(\mathcal X,\mathcal Y)&\le\log_2(2^x+2^y)\\
E(\max(2^\mathcal X,2^\mathcal Y))&\le E(2^\mathcal X+2^\mathcal Y)=E(2^\mathcal X)+E(2^\mathcal Y )\\
&\mbox{z.B.}\mathcal X=10,\,\mathcal Y=5\\
&\log_2(2^{10}+2^{5})
\end{align*}}}\\[.5em]
neue Zufallsvariable
\[\mathcal X_n=2^{T_n} \,\mbox{expotentielle Tiefe}\]
\begin{align*}
E(\mathcal X_n)&=2\sum_{i=1}^n\frac{1}{n}E(\max(\mathcal X_{i-1},\mathcal X_{n-i}))\\
&\le 2\sum_{i=1}^n\frac{1}{n}(E(\mathcal X_{i-1})+E(\mathcal X_n-i))\\
&=\frac{4}{n}\sum_{j=0}^{n-1}E(\mathcal X_j)
\end{align*}
\fbox{\parbox{\columnwidth}{\textbf{Merke:}\[E(\mathcal X_n)=\frac{4}{n}\sum_{i=0}^{n-1}E(\mathcal X_j)\]}}\\[.5em]